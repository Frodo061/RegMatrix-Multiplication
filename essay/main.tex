%----------------------------------------------------------------------------------------
%	PACKAGES AND OTHER DOCUMENT CONFIGURATIONS
%----------------------------------------------------------------------------------------

\documentclass[twoside,twocolumn]{article}

\usepackage{blindtext} % Package to generate dummy text throughout this template 

\usepackage[T1]{fontenc} % Use 8-bit encoding that has 256 glyphs
\usepackage{microtype} % Slightly tweak font spacing for aesthetics
\usepackage[utf8]{inputenc}
\usepackage{indentfirst}

\usepackage[english]{babel} % Language hyphenation and typographical rules

\usepackage[hang, small,labelfont=bf,up,textfont=it,up]{caption} % Custom captions under/above floats in tables or figures
\usepackage{booktabs} % Horizontal rules in tables

\usepackage{enumitem} % Customized lists

\usepackage{abstract} % Allows abstract customization
\renewcommand{\abstractnamefont}{\normalfont\bfseries} % Set the "Abstract" text to bold

\usepackage{titlesec} % Allows customization of titles
\renewcommand\thesection{\Roman{section}} % Roman numerals for the sections
\renewcommand\thesubsection{\roman{subsection}} % roman numerals for subsections
\titleformat{\section}[block]{\large\scshape\centering}{\thesection.}{1em}{} % Change the look of the section titles
\titleformat{\subsection}[block]{\large}{\thesubsection.}{1em}{} % Change the look of the section titles

\usepackage{titling} % Customizing the title section

\usepackage{hyperref} % For hyperlinks in the PDF

%----------------------------------------------------------------------------------------
%	TITLE SECTION
%----------------------------------------------------------------------------------------

\setlength{\droptitle}{-4\baselineskip} % Move the title up

\title{Article Title} % Article title
\author{
João Alves\\[1ex] % Your name
\normalsize University of Minho \\ % Your institution
\normalsize \href{mailto:a77070@alunos.uminho.pt}{a77070@alunos.uminho.pt} % Your email address
\and % Uncomment if 2 authors are required, duplicate these 4 lines if more
Filipe Silva\\[1ex] % Second author's name
\normalsize University of Minho \\ % Second author's institution
\normalsize \href{mailto:jane@smith.com}{jane@smith.com} % Second author's email address
}
\date{} % Leave empty to omit a date
\renewcommand{\maketitlehookd}{%
\begin{abstract}

\blindtext

\end{abstract}
}

%----------------------------------------------------------------------------------------

\begin{document}

% Print the title
\maketitle

%----------------------------------------------------------------------------------------
%	ARTICLE CONTENTS
%----------------------------------------------------------------------------------------

\section{Introduction}

\blindtext % Dummy text

\section{Hardware Platform Characterisation}

\blindtext

\section{PAPI Performance Counters}

The PAPI project successfully implements a cross-platform interface of performance hardware counters that may provide information on how to reduce computational bottlenecks. However, in order to get relevant data from the API to our case study, we had to select a few of these counters. So, to list all the available counters on the 662 nodes (PAPI version 5.5.0), we used \emph{papi\_avail} and chose the following:

\begin{itemize}
    \item \textbf{PAPI\_LD\_INS} Load instructions;
    \item \textbf{PAPI\_L1\_DCM} Level 1 data cache misses;
    \item \textbf{PAPI\_L2\_TCM} Level 2 cache misses;
    \item \textbf{PAPI\_L3\_TCM} Level 3 cache misses;
    \item \textbf{PAPI\_FP\_OPS} Floating point operations;
    \item \textbf{PAPI\_VEC\_SP} Single precision vector/SIMD instructions.
\end{itemize}

Counting these events allows us to induce certain values for important metrics, that affect the algorithm's performance. These are going to be discussed in later sections, but some of them include the miss rate for a certain level of cache, the number of RAM accesses per instruction, the of floating point operations executed and the number of SIMD instructions, which is crucial when we are trying to achieve the maximum performance, on superscalar architectures.

\section{Matrix Dot-Product Algorithm Analysis}

Following the analysis and the characterisation of the hardware environment, the case study that we will be applying these modeling concepts consists of the dot product of two square matrices, where:

$$C = A \times B$$

Being \textbf{A} and \textbf{B} two square matrices of size $\textbf{N} \times \textbf{N}$ and \textbf{C} the result of the dot product between them.

\section{Methods}

Maecenas sed ultricies felis. Sed imperdiet dictum arcu a egestas. 
\begin{itemize}
\item Donec dolor arcu, rutrum id molestie in, viverra sed diam
\item Curabitur feugiat
\item turpis sed auctor facilisis
\item arcu eros accumsan lorem, at posuere mi diam sit amet tortor
\item Fusce fermentum, mi sit amet euismod rutrum
\item sem lorem molestie diam, iaculis aliquet sapien tortor non nisi
\item Pellentesque bibendum pretium aliquet
\end{itemize}
\blindtext % Dummy text

Text requiring further explanation\footnote{Example footnote}.

%------------------------------------------------

\section{Results}

\begin{table}
\caption{Example table}
\centering
\begin{tabular}{llr}
\toprule
\multicolumn{2}{c}{Name} \\
\cmidrule(r){1-2}
First name & Last Name & Grade \\
\midrule
John & Doe & $7.5$ \\
Richard & Miles & $2$ \\
\bottomrule
\end{tabular}
\end{table}

\blindtext % Dummy text

\begin{equation}
\label{eq:emc}
e = mc^2
\end{equation}

\blindtext % Dummy text

%------------------------------------------------

\section{Discussion}

\subsection{Subsection One}

A statement requiring citation \cite{Figueredo:2009dg}.
\blindtext % Dummy text

\subsection{Subsection Two}

\blindtext % Dummy text

%----------------------------------------------------------------------------------------
%	REFERENCE LIST
%----------------------------------------------------------------------------------------

\begin{thebibliography}{99} % Bibliography - this is intentionally simple in this template

\bibitem[Figueredo and Wolf, 2009]{Figueredo:2009dg}
Figueredo, A.~J. and Wolf, P. S.~A. (2009).
\newblock Assortative pairing and life history strategy - a cross-cultural
  study.
\newblock {\em Human Nature}, 20:317--330.
 
\end{thebibliography}

%----------------------------------------------------------------------------------------

\end{document}
